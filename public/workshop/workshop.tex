\documentclass{article}

% if you need to pass options to natbib, use, e.g.:
%     \PassOptionsToPackage{numbers, compress}{natbib}
% before loading neurips_2021

% ready for submission
\usepackage{workshop}

% to compile a preprint version, e.g., for submission to arXiv, add add the
% [preprint] option:
%     \usepackage[preprint]{neurips_2021}

% to compile a camera-ready version, add the [final] option, e.g.:
%     \usepackage[final]{neurips_2021}

% to avoid loading the natbib package, add option nonatbib:
%    \usepackage[nonatbib]{neurips_2021}

\usepackage[utf8]{inputenc} % allow utf-8 input
\usepackage[T1]{fontenc}    % use 8-bit T1 fonts
\usepackage{hyperref}       % hyperlinks
\usepackage{url}            % simple URL typesetting
\usepackage{booktabs}       % professional-quality tables
\usepackage{amsfonts}       % blackboard math symbols
\usepackage{nicefrac}       % compact symbols for 1/2, etc.
\usepackage{microtype}      % microtypography
\usepackage{xcolor}         % colors

\title{Workshop Proposal: Advances in Programming Languages and Neurosymbolic Systems (AIPLANS)}

% The \author macro works with any number of authors. There are two commands
% used to separate the names and addresses of multiple authors: \And and \AND.
%
% Using \And between authors leaves it to LaTeX to determine where to break the
% lines. Using \AND forces a line break at that point. So, if LaTeX puts 3 of 4
% authors names on the first line, and the last on the second line, try using
% \AND instead of \And before the third author name.

\author{%
    David S.~Hippocampus\thanks{Use footnote for providing further information
    about author (webpage, alternative address)---\emph{not} for acknowledging
    funding agencies.} \\
    Department of Computer Science\\
    Cranberry-Lemon University\\
    Pittsburgh, PA 15213 \\
    \texttt{hippo@cs.cranberry-lemon.edu} \\
% examples of more authors
% \And
% Coauthor \\
% Affiliation \\
% Address \\
% \texttt{email} \\
% \AND
% Coauthor \\
% Affiliation \\
% Address \\
% \texttt{email} \\
% \And
% Coauthor \\
% Affiliation \\
% Address \\
% \texttt{email} \\
% \And
% Coauthor \\
% Affiliation \\
% Address \\
% \texttt{email} \\
}

\begin{document}

    \maketitle

    \begin{abstract}
        The development of practical libraries for automatic differentiation has enabled rapid progress in gradient-based learning over the last decade. Other forms of automatic programming now emerging in the statistical learning and programming language community hold the promise of unleashing similar progress in nearby fields, from probabilistic to classical logic. Concurrently, machine learners have made steady progress in representing and synthesizing programs. Other workshops have explored these topics separately, yet few have highlighted the interplay between automatic and synthetic programming, a situation we hope to remedy.
    \end{abstract}

    \section{Introduction}

    Neural information processing systems have benefited tremendously from the development of languages and abstractions for automatic differentiation. Similar domain-specific languages have begun to automate inference in other logical disciplines, such as probabilistic inference, classical logic, and message passing schemes on tree- and graph-structured data.

    Not only does machine learning itself benefit from tools and languages for programmable inference, learning can also be seen as a kind of programming language with a unique intermediate representation, and which is now being used to produce human-readable procedures. Early synthetic functions are now possible thanks to recent progress in statistical langauge modeling and functional programming, resembing procedures a human might plausibly write.

    Using techniques from programmable inference to generate programs, and using insights learned developing those programs to drive innovation in AD and probabilistic programming is a now virtuous cycle, with a steady and growing stream of software and academic papers.

    However many ideas are being reinvented in this process. Automatic differentiation was invented over a half a dozen times in various disciplines over the last century. Functional and type-safe programming are only now being added to Python, the primary language used in machine learning. AD designers continue rediscover connections to implicit differentiation, bilevel optimization, stochastic processes and other fields. Much work remains.

    The duality between code and data for instance is well-known in PL under the guise of homoiconicity. Many other topics that machine learners are just encountering have been well-studied in the programming language community. Programming language designers have thought deeply about higher-order functions, currying and rewriting, and denotational and operational semantics, which enables APIs to compose well and work correctly.

    Similarly, programming language theory has wrestled with issues of expressivity and tractability, and intensional and extensional representation, a distinction which has long since been reconciled by the statistical learning community under the umbrella of model-based learning and approximation theory. Other areas where the interaction could be fruitful are tools for equivalence, proof search and metrics. New language models could enable natural langauge and assitive programming.

    As outlined above, we believe that recent advances in statistical learning and programming languages have been largely siloed, and these two communities have much to learn from each other. Exchanging ideas in a joint workshop could help reveal unrealized connections. Our workshop is designed to be as inclusive as possible. For illustration, we include the following non-exhaustive list of topics:

    \begin{itemize}
      \item Differentiable programming / automatic differentiation
      \item Probabilistic programming / statistical inference
      \item Declarative programming / constraint programming
      \item Dynamic programming / reinforcement learning
      \item Functional programming / $\lambda$-calculus
      \item Array programming / linear algebra
      \item Semiring programming / message passing
      \item Metaprogramming / reflection
      \item Logic programming / proof search
      \item Domain-specific languages
    \end{itemize}

    We also encourage developers of libraries and frameworks to submit their work for evaluation.

\end{document}